\subsection{Lagrangian in Differentialformen}

Um den Vortrag abzurunden, kommen wir wieder kurz auf die Lagrange-Beschreibung zurück und zeigen, wie man die Maxwell-Gleichungen in Differentialformen durch einen Lagrangian herleitet. Dazu muss man nun auch den Lagrangian in Differentialformen ausdrücken. \\
Derartige Ausdrücke werden später auch in der Yang-Mills-Theorie un der Chern-Simon-Theorie auftreten. Ihr Vorteil ist, dass sie automatisch alle geforderten Punkte in Kapitel 1 erfüllen, insbesondere die Koordinatenunabhängigkeit.
