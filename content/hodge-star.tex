\subsection{Hodge-Stern-Operator}
Der Hodge-Stern-Operator ist eine der drei Standard-Operation auf dem Raum der Differentialformen. \\
Wir betrachten die Abbildung:
\begin{align*}
\star : \Lambda^p(V*) \quad \rightarrow \quad  \Lambda^{(n-p)}(V*) 
\end{align*}

Die abstrakte Definition des Operators lautet: \todo{Hier mal noch die Definition überprüfen}

\begin{align}
\alpha \w \star \beta = g(\alpha,\beta) \omega 
\end{align}
 
 für alle  $\alpha, \beta \in \Lambda^p V$ und  $\omega \in \Lambda^n V $.\\
 Die Metrik g definiert sich für die geordnete Orthonormalbasis {$\sigma^{i} \in V. i=1,\dots,n$} wie folgt:

 \begin{align}
g(\alpha,\beta) = \text{det}\left( g(\alpha^{i}, \beta^{i})\right) \text{mit } g(\sigma^{i},\sigma^{j}) = \pm \delta^{ij}
 \end{align}
  für alle "einfachen" p-Vektoren $\alpha$ und $\beta$. \\
Im Allgemeinen gilt:
 \begin{align}
 g(\omega,\omega) = \prod_{k=1}^{n} g(\sigma^{i},\sigma^{i}) = (-1)^s
 \end{align}
 mit der Signatur $s$, welche sich aus der Anzahl an Minuszeichen bei der Betrachtung der Metrik ergibt und die Norm des Vektors festlegt. \\
 In unserem Fall bietet sich die nachfolgende, etwas weniger abstrakte, praktische Definition an. \\
 Wir betrachten einen Basis-p-Vektor $\alpha$ der Form:
 
 \begin{align}
 \alpha= \sigma^{1} \w \sigma^{2} \w \dots \w \sigma^{p} = \sigma^{I} \in \Lambda^p V
 \end{align}
 
 Einsetzen von $\sigma^{J} \in \Lambda^{(n-p)}V$ in die abstrakte Definition liefert:
 

\begin{align}
\sigma^{J} \w \star \alpha= g(\sigma^{j},\alpha) \omega 
\end{align}

Daraus folgt direkt: \quad $\star\alpha = \text{const.} \cdot \sigma^{p+1} \w \dots \w \sigma^{n}$ \\
und mit $\sigma^{J} \w\star\alpha = \omega$: \quad  $1 =  g(\sigma^{j},\text{const.} \cdot \sigma^{j})$, \\
was uns schließlich zum Ausdruck:
\begin{align*}
\text{const.}= \frac{(-1)^s}{g(\sigma^j,\sigma^j)} = \frac{g(\omega,\omega)}{g(\sigma^j,\sigma^j)} = g(\alpha,\alpha)
\end{align*}
führt. \\
Alles zusammen ergibt uns die recht anschauliche Definition:
\begin{mybox}{Hodge-Stern-Operator}
\begin{align}
\star(\sigma^1 \w \sigma^2 \w \dots \w \sigma^p) = g(\sigma^1,\sigma^1)\dots g(\sigma^p,\sigma^p) \sigma^{p+1}\w \dots \w \sigma^n
\end{align}
\end{mybox}

Für die zweifache Anwendung des Operators gilt: (ohne Herleitung)
\begin{align}
\star \star = (-1)^{p(n-p)+s}
\end{align}

\subsubsection{Anwendungsbeispiel: 4D-Minkowski-Raumzeit}

Wir betrachten den trivialen 4er-Vektor:

\begin{align*}
\omega = - \dd t \w \dd x\w \dd y \w \dd z
\end{align*}

Wenden wir nun den $\star$-Operator an folgt zum Beispiel:

\begin{itemize}
\begin{align*}
\item \star 1 &= \omega \\
\item \star \omega &= -1 \\
\item  \star \dd t &= - \dd x \w \dd y \w \dd z \\
\item \star \dd x \w \dd y \w \dd z &= - \dd t \\
\item \star\star \omega &= (-1)^{4(4-4)+1} \omega = -\omega \\
\item \dots
\end{align*}
\end{itemize}

Wörtlich gesprochen liefert uns der Hodge-Stern-Operator die duale (n-p)-Form  zur p-Form durch "wedgen" der übrigen Basis-1-Formen, die nicht in der p-Form auftreten, unter Berücksichtigung der Geometrie des Raumes. \\
Dies wollen wir nutzen, um nun die inhomogenen Maxwell-Gleichungen neu auszuformulieren.