\subsection{Lagrangian in Differentialformen}

Um den Vortrag abzurunden, kommen wir wieder kurz auf die Lagrange-Beschreibung zurück und zeigen, wie man die Maxwell-Gleichungen in Differentialformen durch einen Lagrangian herleitet. Dazu muss man nun auch den Lagrangian in Differentialformen ausdrücken. \\
Derartige Ausdrücke werden später auch in der Yang-Mills-Theorie un der Chern-Simon-Theorie auftreten. Ihr Vorteil ist, dass sie automatisch alle geforderten Punkte in Kapitel 1 erfüllen, insbesondere die Koordinatenunabhängigkeit. \\

Die Lagrangians setzt man meistens axiomatisch an und zeigt dann, dass sie die gewünschten Gleichungen liefern. Für das elektromagnetische Feld gilt:

\begin{align}
\lagr = - \frac{1}{2} F F + A J,
\end{align}

wobei F, A und J gemäß den obigen Definitionen gegeben sind. Es gelte $F = \d A$. \\

Das zugehörige Wirkungsfunktional ist $\text{I[A]}=\int \lagr$ und sei $A + \varepsilon \cdot \delta A$ das gestörte Eichpotential. \\
Dann folgt aus der Variation der Wirkung: 

\begin{align*}
