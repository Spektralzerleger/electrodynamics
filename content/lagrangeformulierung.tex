\section{Lagrange-Formulierung der Elektrodynamik}
Das Hamilton'sche Extremalprinzip ist in vielen Anwendungen sehr erfolgreich. Es charakterisiert physikalisch realisierbare Bahnen unter allen denkbaren Bahnen als diejenigen, die kritische Elemente des Wirkungsintegrals sind. Die Lagrangefunktion dient nicht nur zur rationalen Herleitung der Bewegungsgleichungen, sondern ist auch ein wichtiges Hilfsmittel zur Bestimmung von Erhaltungsgrößen.
Wir werden diese mächtige Methode auf die Elektrodynamik anwenden und sie auf überabzählbar viele Freiheitsgrade erweitern.

\subsection{Lagrangedichte für eine Feldtheorie}
In der klassischen Mechanik hatten wir es immer mit endlich vielen Freiheitsgraden zu tun. Bei der klassischen Feldtheorie stellen wir uns den Raum als gitterartig verbundene Raumpunkte vor, die jeweils mit ihren Nachbarn wechselwirken können. Im Kontinuumslimit haben wir unendlich viele Raumpunkte und benötigen eine neue Beschreibungsweise. \\
An die Stelle der verallgemeinerten Koordinaten treten nun die \textbf{Felder} $\varphi^{(i)}$ und die Lagrangefunktion wird durch die \textbf{Lagrangedichte} \  $\text{L}=\iiint \dd[3]{x}  \lagr$ \ ersetzt. \\
Das Wirkungsintegral ist nun $\mathcal{S}=\int_{t_1}^{t_2} \dd{t} \iiint \dd[3]{x}  \lagr = \int \dd[4]{x} \lagr$. \\
Die Bewegungsgleichungen erhalten wir aus dem Hamilton'schen Prinzip und nehmen dazu an, dass die Variation der Felder auf den Hyperflächen $t=t_1$ und $t=t_2$ verschwindet:
\begin{align*}
0 = \delta \mathcal{S} &= \int_{t_1}^{t_2} \dd t \iiint \dd[3]{x} \left( \dphi \delta\varphi^{(i)} + \ddphi \delta(\partial_{\mu}\varphi^{(i)}) \right) \\
 				 &= \int_{t_1}^{t_2} \dd t \iiint \dd[3]{x} \left(\dphi - \partial_{\mu} \ddphi \right) \delta\varphi^{(i)}
\end{align*}
Wir erhalten daraus:
\begin{mybox}{Euler-Lagrange-Gleichung}

\begin{align}
0=\dphi - \partial_{\mu} \ddphi
\end{align}

\end{mybox}

\subsection{Lagrangedichte für das Maxwell-Feld}
Wir wollen $\lagr$ \ so wählen, dass sie ein Skalar ist, von den Feldern sowie deren Ableitungen abhängt, Strom/Ladung als Quelle hat und die Maxwell-Gleichungen als Bewegungsgleichungen liefert:
\begin{align}
\partial_{\mu}\fu = j^{\nu} \qquad  \Longleftrightarrow \qquad \Box A^{\nu} - \partial^{\nu}(\partial_{\mu}A^{\mu}) = j^{\nu}
\end{align}
Da wir als Bewegungsgleichungen Differentialgleichungen 2. Ordnung erwarten, empfiehlt es sich, das 4er-Feld $A^{\mu}$ wie folgt anzusetzen:
\begin{align}
\lagr = \text{a} \cdot \partial_{\mu}A^{\nu}\partial^{\mu}A_{\nu} + \text{b} \cdot \partial_{\mu}A^{\nu}\partial_{\nu}A^{\mu} + \text{c} \cdot (\partial_{\mu}A^{\mu})^2 + \text{d} \cdot A_{\mu}A^{\mu} + \text{e} \cdot A_{\mu}j^{\mu}
\end{align}
Setzt man dies in (1) ein und wählt die Koeffizienten so, dass (2) herauskommt, findet man:
\begin{align}
\lagr = -\frac{1}{4} \fd \fu - j_{\mu}A^{\mu} + \frac{\text{c}}{2} \underbrace{\left[ (\partial_{\mu}A^{\mu})^2 - \partial_{\mu}A^{\nu}\partial_{\nu}A^{\mu} \right]}_{= \partial_{\mu}(A^{\mu}\partial_{\varrho}A^{\varrho}-A^{\nu}\partial_{\nu}A{\mu})}
\end{align}
Da der letzte Term sich als Divergenz schreiben lässt, ist er nur ein Randterm und verschwindet bei der Variation. Er hat somit keine Auswirkungen auf die  Bewegungsgleichungen! \\ 
Es folgt also:
\begin{mybox}{Lagrangedichte}
\begin{align}
\lagr = -\frac{1}{4} \fd \fu - j_{\mu}A^{\mu}
\end{align}
\end{mybox}
\phantom{.} \\
Die Lagrangedichte hat demnach die richtige Dimension $\left[\sfrac{\text{Energie}}{\text{Volumen}}\right]$, wobei der erste Term einer "kinetischen Energie" und der Zweite einer "potentiellen Energie" entspricht. \\

Andererseits kann man die Lagrangedichte auch ohne Kenntnis der Maxwell-Gleichungen herleiten: \\
Wir wissen, dass die Elektrodynamik eine relativistische Theorie ist und die Bewegungsgleichungen daher invariant unter Lorentz-Transformationen sein sollten. Außerdem gilt in erster Ordnung das Superpositionsprinzip, d.h. Summen von Lösungen der Bewegungsgleichungen sind wieder Lösungen. Deshalb sollten die Bewegungsgleichungen linear sein, der Lagrangian also höchstens quadratisch in den Ableitungen der Feldkomponenten sein. Es könnten auch Terme höherer Ordnung auftreten, diese wären aber nur kleine Störterme, die die Bewegungsgleichungen nur unwesentlich beeinflussen. \\ 
Zuletzt sollen die Bewegungsgleichungen eichinvariant sein.
Aus den obigen Überlegungen ergeben sich folgende Forderungen an den Lagrangian: 
\begin{enumerate}
	\item Lorentzinvarianz
	\item Eichinvarianz 
\end{enumerate}

Aus der ersten Forderung folgt, dass die Lagrangedichte ein Lorentz-Skalar sein muss. Aus der Eichinvarianz folgt, dass die Lagrangedichte $\fd$ oder sein Pendant $\star \fd$ enthalten kann, denn unter einer Eichtransformation $A^{\mu} \rightarrow A^{\mu} + \partial^{\mu} \phi$ ändert sich der Feldstärketensor nicht:
\begin{align}
\fd ' = \partial_{\mu} \left( A^{\nu} + \partial_{\nu} \phi \right) - \partial_{\nu} \left( A^{\mu} + \partial_{\mu} \phi \right) = \partial_{\mu}  A^{\nu} - \partial_{\nu} A^{\mu} = \fd.
\end{align}

Da $\fd$ spurfrei ist, sind die ersten möglichen eichinvarianten Skalare:
\begin{align}
\fd \fu &= - \left( \star \fd \right) \left( \star \fu \right) = 2 \left( \vec{B}^2 - \vec{E}^2 \right) \\
\left( \star \fd \right) \fu &= 4 \vec{B} \cdot \vec{E}
\end{align}

	Betrachtet man diese beiden Skalare ausführlicher, fällt auf, dass der Zweite sein Vorzeichen unter Raumspiegelung ändert. Solche paritätsverletzenden Effekte wurden bisher nicht beobachtet, weshalb dieser als mögliche Option nicht in Betracht kommt. Außerdem lässt er sich sowieso als Divergenz schreiben und liefert demnach keinen Beitrag bei der Variation. Da wir höchstens quadratische Abhängigkeiten in den Ableitungen zulassen wollen, haben wir diese mit $\fd \fu$ gefunden. Betrachten wir also noch die reinen Feldanteile: \\

Am intuitivsten wäre $A_{\mu} A^{\mu}$. Dies entspricht aber einem Massenterm, der dazu führen würde, dass die Felder schneller als $\frac{1}{r}$ abfallen würden und ist darüber hinaus nicht eichinvariant. Deshalb bleibt noch der Lorentz-Skalar $j_{\mu} A^{\mu}$, der mit der Kontinuitätsgleichung $\partial_{\mu} j^{\mu} = 0$ eichinvariant ist. Da es die Symmetrie der Natur verletzen würde, dürfen die Raumzeitkoordinten $x^{\mu}$ nicht explizit vorkommen. 
Die drei Forderungen fixieren die Lagrangedichte demnach auf $\lagr = a \cdot \fd \fu + b \cdot j_{\mu} A^{\mu}$. Der Faktor $a$ bestimmt die Normierung des $A$-Feldes und wird üblicherweise zu $a = - \frac{1}{4}$ festgelegt. Das negative Vorzeichen rührt daher, dass der 'kinetische Anteil' mit den Zeitableitungen positiv wird, wie es bei der klassischen Lagrangefunktion üblich ist.
Der Faktor $b$ beschreibt die Stärke der Wechselwirkung von $A_{\mu}$ mit dem Strom $j^{\mu}$ und entspricht einer Ladung. Er bleibt als empirisch zu bestimmende Größe stehen. Man erhält im Endeffekt aber wieder (5).
