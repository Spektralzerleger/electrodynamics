\subsection{Eichtransformationen und Eichinvarianz}
Die Eichtransformation $A_{\mu} \rightarrow A'_{\mu} = A_{\mu} - \partial_{\mu}\phi$ \ ändert die Lagrangedichte im Allgemeinen:
\begin{align}
\lagr'(A'_{\tau},\partial_{\tau}A'_{\tau}) = \lagr(A_{\tau},\partial_{\tau}A_{\tau})+ j^{\mu}\partial_{\mu}\phi
\end{align}
Da die Kontinuitätsgleichung $\partial_{\mu}j^{\mu}=0$ gilt, kann man $j^{\mu}\partial_{\mu}\phi$ durch $\partial_{\mu}(j^{\mu}\phi)$ ersetzen.
Dies lässt sich wieder als Divergenz auffassen und hat deshalb keinen Einfluss auf die Bewegungsgleichungen:

\begin{align*}
\text{Ladungserhaltung} \qquad \longleftrightarrow \qquad \text{Eichinvarianz}
\end{align*}
\\
Besonders interessant sind Translationen $x \rightarrow x + \varepsilon \cdot a$: 
\begin{align*}
\varphi^{(i)}(x + \varepsilon \cdot a) &= \varphi^{(i)}(x) + \overbrace{\varepsilon\cdot a^{\mu}\partial_{\mu}\varphi^{(i)}(x)}^{\delta\varphi^{(i)}} \\
\partial_{\mu}\varphi^{(i)}(x + \varepsilon \cdot a) &= \partial_{\mu}\varphi^{(i)}(x) + \underbrace{\varepsilon\cdot\left(a^{\nu}\partial_{\nu}\partial_{\mu}\varphi^{(i)}(x) + \partial_{\mu}a^{\nu}\partial_{\nu}\varphi^{(i)}(x)\right)}_{\delta(\partial_{\mu}\varphi^{(i)})} 
\end{align*}
Es wurde hierbei angenommen, dass die Ortsabhängigkeit nur in den Feldern steckt. \\
Wir erhalten:
\begin{align*}
\delta \mathcal{S} &= \int \dd[4]{x} \left( \dphi \ \varepsilon\cdot a^{\mu}\partial_{\mu}\varphi^{(i)} + \ddphi \ \varepsilon\cdot\left(a^{\nu}\partial_{\nu}\partial_{\mu}\varphi^{(i)}+\partial_{\mu}a^{\nu}\partial_{\nu}\varphi^{(i)}\right) \right) \\
&= \int \dd[4]{x} \left( \partial_{\nu}\lagr - \partial_{\mu} \left[\ddphi \partial_{\nu}\varphi^{(i)}\right] \right) \varepsilon\cdot a^{\nu}
\end{align*}
Damit erhalten wir:
\begin{mybox}{Energie-Impuls-Tensor}
\begin{align}
\partial_{\mu}\underbrace{\left[ \ddphi \partial^{\nu}\varphi^{(i)} - g^{\mu \nu}\lagr \right]}_{ = \eitens} = 0
\end{align}
\end{mybox}

Für das elektromagnetische Feld $\lagr = -\frac{1}{4} \fd \fu - j_{\mu}A^{\mu}$ folgt:

\begin{align}
\eitens = - \fu j^{\nu}A_{\sigma}+g^{\mu\nu}\left( \frac{1}{4} F_{\sigma\tau}F^{\sigma\tau}+j_{\mu}A^{\mu}\right)
\end{align}

In dieser Fom ist der Energie-Impuls-Tensor aber nicht eichinvariant, denn unter der Transformation $A^{\mu} \ \rightarrow \ A^{\mu} + \partial^{\mu}\phi$ folgt:

\begin{align}
\eitens \quad \rightarrow \quad \eitens + g^{\mu\nu}j_{\tau}\partial^{\tau}\phi-F^{\mu\sigma}\partial^{\nu}\partial_{\sigma}\phi = \eitens + \partial_{\sigma}(g^{\mu\nu}j^{\sigma}\phi - F^{\mu\sigma}j^{\nu}\phi) - j^{\mu}\partial^{\nu}\phi
\end{align}

Wir wollen einen Ausdruck für $\eitens$ finden, der eichinvariant und symmetrisch ist. \\
Dazu bemerken wir, dass zur Lagrangedichte eine Divergenz addiert werden kann ohne die Bewegungsgleichungen zu ändern. Also hat der Energie-Impuls-Tensor eine gewisse Beliebigkeit, die wir ausnutzen können. \\
Der Energiestrom bleibt unverändert unter einer Transformation $\eitens \ \rightarrow \ \eitens+ \grad{\eitens}$ mit: 

\begin{enumerate}
\item[(i)] $\partial_{\mu}\grad{\eitens}=0$ \qquad (erfüllt Erhaltungssatz)
\item[(ii)] $ \int \dd[3]{x} \grad{T^{00}}$ \qquad \ (trägt nicht zur Gesamtenergie bei)
\end{enumerate}

Obiger Zusatzterm nach der Eichung motiviert $\grad{\eitens}=\partial_{\sigma}(F^{\mu\sigma}A^{\nu})$ und erfüllt die Bedingungen. \\
Setzen wir also:

\begin{align}
\eitens = -F^{\mu\sigma}\partial^{\nu}A_{\sigma} + g^{\mu\nu}\left( \frac{1}{4}F_{\sigma\tau}F^{\sigma\tau} + j_{\tau}A^{\tau} \right) + \partial_{\sigma} \left(F^{\mu\sigma}A^{\nu} \right)
\end{align}

und schreiben mit der Maxwell-Gleichung $\partial_{\sigma}F^{\mu\sigma}=-j^{\mu}$ den neuen Ausdruck:

\begin{align}
\eitens = F^{\mu\varrho}F_{\varrho}^{\phantom{\varrho}\nu} + \frac{1}{4}g^{\mu\nu}F_{\sigma\tau}F^{\sigma\tau} + g^{\mu\nu}j_{\sigma}A^{\sigma} - j^{\mu}A^{\nu}
\end{align}

Damit gilt $\partial_{\mu}\eitens = A_{\varrho}\partial^{\nu}j^{\varrho}$. \\
Für den rein elektromagnetischen Anteil $\eitens_0 = \fu F_{\sigma}^{\phantom{\sigma}\nu} + \frac{1}{4}g^{\mu\nu}F_{\sigma\tau}F^{\sigma\tau}$ folgt dann: $\partial_{\mu}\eitens_0=j_{\varrho}F^{\varrho\nu}$. \\
\vspace{1pt} \\
In Abwesenheit von äußeren Quellen ist der Energie-Impuls-Tensor \textbf{eichinvariant}, \textbf{erhalten}, \textbf{symmetrisch} und \textbf{spurfrei}.
