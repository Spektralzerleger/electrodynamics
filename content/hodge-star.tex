\subsection{Hodge-Stern-Operator}
Der Hodge-Stern-Operator ist eine der drei Standard-Operation auf dem Raum der Differentialformen. \\
Man spricht beim Kreuzprodukt von einer Rotation von Pseudovektoren, was nun anhand der Definition des Operators veranschaulicht werden soll. Wir werden sehen, dass die "Rechte-Hand-Regel" in die Metrik eingeht.
Wir betrachten eine Abbildung:
\begin{align*}
\star : \Lambda^p(V*) \quad \rightarrow \quad  \Lambda^{(n-p)}(V*) 
\end{align*}
Die abstrakte Definition des Operators lautet:
\begin{align}
\omega \w \star \mu = \left<\omega,\mu \right>\text{vol} = g(\omega,\mu)\alpha \quad \forall \omega, \mu \in \Omega^p(\mathcal{M}), \alpha \in \Omega^n(\mathcal{M})
\end{align}
Dabei ist $\text{vol} = \sqrt{\abs{\text{det} \ g_{\mu\nu}}} \ \dd x^1 \w \dots \w \dd x^n$ die Volumen-Form und $g_{\mu\nu}$ die Metrik. \\

Im Allgemeinen gilt:
 \begin{align}
 g(\omega,\omega) = \prod_{k=1}^{n} g(\sigma^{i},\sigma^{i}) = (-1)^s
 \end{align}
 mit der Signatur $(s,n-s)$, welche die Norm des Vektors festlegt. Die $\sigma^{i}$ sind dabei die orthonormalen Basisvektoren \\
 Für die zweifache Anwendung des Operators gilt: (ohne Herleitung)
 \begin{align}
 \star \star = \star^2 = (-1)^{p(n-p)+s}
 \end{align}
 Wir wollen mit einer etwas anschaulicheren, praktischen Definition arbeiten, welche sich durch kleinere Umformungen aus der abstrakten Definition ergibt. Wir erhalten:
\begin{mybox}{Hodge-Stern-Operator}
\begin{align}
\star(\sigma^1 \w \sigma^2 \w \dots \w \sigma^p) = g(\sigma^1,\sigma^1)\dots g(\sigma^p,\sigma^p) \sigma^{p+1}\w \dots \w \sigma^n
\end{align}
\end{mybox}

\subsubsection{Anwendungsbeispiel: 4D-Minkowski-Raumzeit}

Wir betrachten den trivialen 4er-Vektor:

\begin{align*}
\omega = - \dd t \w \dd x\w \dd y \w \dd z
\end{align*}

Wenden wir nun den $\star$-Operator an folgt zum Beispiel:

\begin{itemize}
\begin{align*}
\item \star 1 &= \omega \\
\item \star \omega &= -1 \\
\item  \star \dd t &= - \dd x \w \dd y \w \dd z \\
\item \star \dd x \w \dd y \w \dd z &= - \dd t \\
\item \star\star \omega &= (-1)^{4(4-4)+1} \omega = -\omega \\
\item \dots
\end{align*}
\end{itemize}

Wörtlich gesprochen liefert uns der Hodge-Stern-Operator die duale (n-p)-Form  zur p-Form durch "wedgen" der übrigen Basis-1-Formen, die nicht in der p-Form auftreten, unter Berücksichtigung der Geometrie des Raumes. \\
Dies wollen wir nutzen, um nun die inhomogenen Maxwell-Gleichungen neu auszuformulieren.