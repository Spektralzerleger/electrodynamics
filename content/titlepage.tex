\begin{center}

	\makeatletter
	\thispagestyle{plain}
	\LARGE\textbf{\@title} \\
	\vspace{2mm}
	\large\bfseries{\@author} \\
	\normalfont
	\vspace{2mm}
	\large{\@date} \\
	\vspace{2mm}
	\large{Institut für Theoretische Physik \\
		Universität Heidelberg} \\
	\makeatother
\end{center}

\normalsize

Dieser Vortrag entstand im Rahmen des Seminars "\textit{Klassische Elektrodynamik}", organisiert von Prof. Weigand am Institut für Theoretische Physik der Universität Heidelberg im Sommersemester 2018. \\
Ziel des Vortrages ist es, eine Formulierung der Elektrodynamik in Differentialformen zu motivieren und damit dann eine alternative Formulierung der Maxwell-Gleichungen einzuführen. Es werden die mathematischen und physikalischen Grundlagen für das Verständnis der modernen Eichtheorien besprochen und die Elektrodynamik als klassische Feldtheorie vorgestellt. \\
Zu Beginn soll die Lagrange-Formulierung der kovarianten Elektrodynamik thematisiert werden, um sie am Ende in der Sprache der Differentialformen zu formulieren.
