\subsection{Maxwell-Gleichungen}

% Anfang fehlt noch ..zuerst mal die Überleitung zum Hodge

Auf Grundlage der elektromagnetischen Dualität, welche durch die Transformationen der Felder gemäß
\begin{align*}
E \rightarrow B \quad sowie \quad B \rightarrow -E
\end{align*}

die Invarianz der Maxwell-Gleichungen im Vakuum charakterisiert, wollen wir nun eine neue Methode, genauer einen neuen Operator einführen der uns diese Transformationen ermöglicht. Wir erinnern uns an die neue Darstellung der Felder nach Gl. ("wird nach passend eingefügt") und ("wird nach passend eingefügt"):

\begin{align*}
E &= E_x \dd x +E_y \dd y + E_z \dd z   \\
B &= B_x \dd y \w \dd z + B_y \dd z \w \dd x + B_z \dd x \w \dd y 
\end{align*}

Für die Transformation müssen wir also eine 1-Form in eine 2-Form überführen und umgekehrt. Im $\mathbb{R}^3$ kann diese Transformation mit dem \bfseries Hodge-Stern-Operator \normalfont realisiert werden.