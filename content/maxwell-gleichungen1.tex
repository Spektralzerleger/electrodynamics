\subsection{homogene Maxwell-Gleichungen}
Wir wollen zunächst einmal die beiden homogenen Maxwell-Gleichungen umformulieren. \\
Dazu nehmen wir einen allgemeinen Standpunkt ein und müssen das elektrische und das magnetische Feld als Vektorfelder ind der 4D-Raumzeit auffassen. Wir arbeiten nachfolgend in der Minkowski-Raumzeit in den Standard-Raumzeit-Koordinaten $x^{\mu}$. \\
Zunächst einmal bemerken wir, dass sich die äußere Ableitung einer 1-Form in einen Raumanteil und einen Zeitanteil aufteilen lässt:

\begin{align}
\dd \omega &= \partial_0 \omega_I \dd x^0 \w \dd x^{I} + \partial_i \omega_I \dd x^i \w \dd x^{I} \\
&= \dd t \w \partial_t \omega + \dd_s \omega 
\end{align}

wobei I über alle Multi-Indizes läuft. \\
Betrachtet man die Gleichungen (1) und (2), so fällt auf, dass es sinnvoll erscheint das E-Feld als 1-Form sowie das B-Feld als 2-Form zu schreiben:

\begin{align}
E &= E_x \dd x +E_y \dd y + E_z \dd z   \\
B &= B_x \dd y \w \dd z + B_y \dd z \w \dd x + B_z \dd x \w \dd y 
\end{align}

Damit lassen sich (1) und (2) schreiben als:

\begin{align}
\dd_s B &= 0 \\
\partial_t B + \dd_s E &=0
\end{align}

Wir wollen diese beiden Gleichungen zusammenfassen, indem wir für das elektromagnetische Feld die 2-Form $F$ im $\mathbb{R}^4$ einführen:

\begin{align}  
F= E \w \dd t + B 
\end{align} 

denn dann haben wir:

\begin{align}
\dd F &= \dd E \w \dd t + \dd B \notag \\
		&= (\dd_s E + \dd t \w \partial_t E ) \w \dd t + \dd_s B + \dd t \w \partial_t B \notag \\
		&= (\dd_s E + \partial_t B) \w \dd t + \dd_s B
\end{align}

Damit lassen sich beide homogenen Maxwell-Gleichungen kurz zu: 

\begin{align}
\dd F = 0
\end{align}

zusammenfassen. \\
In Komponentenschreibweise gilt für das elektromagnetische Feld

\begin{align}
F = \frac{1}{2} \fd \dd x^{\mu} \w \dd x^{\nu}
\end{align}

mit dem Feldstärke-Tensor:
\begin{align}
\fd =
\begin{pmatrix}
\phantom{-}0 & -E_x & -E_y & -E_z \\
\phantom{-}E_x & \phantom{-}0 & \phantom{-}B_z & -B_y \\
\phantom{-}E_y & -B_z & \phantom{-}0 & \phantom{-}B_x \\
\phantom{-}E_z & \phantom{-}B_y & -B_x & \phantom{-}0
\end{pmatrix}
\end{align}

Da die äußere Ableitung koordinatenunabhängig ist, gilt $\dd F = 0$ auf jeder beliebigen Mannigfaltigkeit und hängt nicht von der Wahl der Koordinaten ab. \\
Die 2-Form $F$ ist demnach geschlossen und besitzt nach dem \bfseries Poincaré-Lemma \normalfont ein Potential:

\begin{align}
F = \dd A 
\end{align}

wobei A eine 1-Form ist. \\
Damit gilt automatisch $ \dd F = \dd (\dd A) = 0$. Außerdem sehen wir, dass wir eine Eichfreiheit haben:

\begin{align}
A \quad \rightarrow \quad A + \dd \lambda \qquad\qquad \lambda \in C^{\infty}(\mathbb{R}^4)
\end{align}.

Mit Hilfe dieses Potentials lassen sich die Felder wie folgt darstelllen:

\begin{align}
E &= -\partial_t A \\
B &= \dd_s A
\end{align}

Betrachten wir nun noch die beiden inhomogenen Gleichungen (3) und (4), die jeweils noch Quellterme enthalten. Es fällt auf, dass $E$ und $B$ gewissermaßen ihre Rollen tauschen. Dieses Phänomen wird als \bfseries elektromagnetische Dualität \normalfont bezeichnet.

Auf Grundlage dieser Dualität, welche durch die Transformationen der Felder gemäß
\begin{align*}
E \rightarrow B \qquad \text{sowie} \qquad B \rightarrow -E
\end{align*}

die Invarianz der Maxwell-Gleichungen im Vakuum charakterisiert, wollen wir nun eine neue Methode, genauer einen neuen Operator einführen der uns diese Transformationen ermöglicht. \\
Wir erinnern uns an die neue Darstellung der Felder nach Gl. (\todo{richtige Nummer}) und (\todo{richtige Nummer}):
\begin{align*}
E &= E_x \dd x +E_y \dd y + E_z \dd z   \\
B &= B_x \dd y \w \dd z + B_y \dd z \w \dd x + B_z \dd x \w \dd y 
\end{align*}
<<<<<<< HEAD
Für die Transformation müssen wir also eine 1-Form in eine 2-Form überführen und umgekehrt. Im $\mathbb{R}^3$ kann diese Transformation mit dem \bfseries Hodge-Stern-Operator \normalfont realisiert werden.
=======

Für die Transformation müssen wir also eine 1-Form in eine 2-Form überführen und umgekehrt. Im $\mathbb{R}^3$ kann diese Transformation mit dem \bfseries Hodge-Stern-Operator \normalfont realisiert werden.
>>>>>>> 8b7201361530f31e0939f488e9005b8929dd4179
