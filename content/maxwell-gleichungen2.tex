\subsection{inhomogene Maxwell-Gleichungen}

Bei den inhomogenen Maxwell-Gleichungen tauschen nicht nur die Rollen von $E$- und $B$-Feld, es treten zusätzlich Quellterme in Form der Ladungsdichte $\rho$ und der Stromdichte $\vec{j}$ auf. \\
Um diese als Differentialformen zu betrachten, nutzen wir, dass die Metrik uns erlaubt Vektorfelder in 1-Formen umzuwandeln. Wir definieren also  

\begin{align}
j = j_1 \dd x^1 + j_2 \dd x^2 + j_3 \dd x^3. 
\end{align}

Analog kann man das 4-Vektorfeld $j^{\mu}$ mit der Minkowski-Metrik in eine Differentialform umwandeln:

\begin{align}
J = j - \rho \dd t. 
\end{align}

Nun können wir die inhomogenen Maxwell-Gleichungen in der Sprache der Differentialformen schreiben als:

\begin{align}
\star_s \dd_s \star_s E &= \rho \\
\star_s \dd_s \star_s B - \partial_t E &= j
\end{align}

Diese beiden Gleichungen lassen sich wieder zu einer einzigen zusammenfassen, wenn man bemerkt, dass:

\begin{align}
\star F &= \star_s E - \star_s B \w \dd t \\
\dd \star F &= \star_s \partial_t E  \w \dd t + \dd_s \star_s E - \dd_s \star_s B \w \dd t \\
\star \dd \star F &= - \partial_t E - \star_s \dd_s \star_s E \w \dd t + \star_s \dd_s \star_s B 		
\end{align}

Setzt man $\star \dd \star F = J$ und vergleicht Terme gleichen Grades erhält man das gewünschte Ergebnis: Gleichungen () und () in einer! Also lassen sich die inhomogenen Maxwell-Gleichungen in Differentialformen schreiben als $\star \dd \star F = J$ oder äquivalent durch nochmalige Anwendung des Hodge-Stern-Operators auf beiden Seiten:

\begin{align}
\dd \star F = \star J		
\end{align}


Insgesamt lässt sich also die gesamte Elektrodynamik auf die beiden gefundenen, äußerst eleganten Gleichungen zurückführen.

\begin{mybox}{Maxwell-Gleichungen}
\begin{align*}
\dd F &= 0 \\
\dd \star F &= \star J
\end{align*}
\end{mybox}
