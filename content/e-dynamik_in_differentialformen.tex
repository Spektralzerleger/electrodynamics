\section{Elektrodynamik in Differentialformen}
Wir wollen die Maxwell-Gleichungen

\begin{align}
\div{\vec{B}} &= 0 \\
\curl \vec{E} + \partial_t \vec{B} &= 0 \\
\div{\vec{E}} &= \varrho \\
\curl \vec{B} - \partial_t \vec{E} &= \vec{j}
\end{align} 

in der Sprache der Differentialformen umformulieren. Dies wirft einerseits Licht auf einige ihrer Eigenschaften, die bei einer analogen Behandlung im Rahmen der Vektoranalysis nicht so deutlich hervortreten und bringt andererseits den geometrischen Charakter der Gleichungen noch deutlicher hervor. Außerdem bereitet es die Grundlage für das Verständnis der nicht-Abel'schen Eichtheorien, die für die Beschreibung der fundamentalen Wechselwirkungen der Natur wesentlich sind.