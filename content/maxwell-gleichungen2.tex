\subsection{inhomogene Maxwell-Gleichungen}
Durch Anwenden des Hodge-Stern-Operators auf $F$ folgt:

\begin{align}
\star F = \star_s E - \star_s B \w \dd t 
\end{align}

also entspricht die Dualität der Transformation 

\begin{align*}
B \rightarrow \star_s E \qquad \text{sowie} \qquad E \rightarrow - \star_s B
\end{align*}

In Komponentenschreibweise sehen wir, dass:




\begin{align}
\dd \star F &= \star_s \partial_t E  \w \dd t + \dd_s \star_s E - \dd_s \star_s B \w \dd t \notag \\
				&= \dd_s \star_s E + (\star_s \partial_t E - \dd_s \star_s B) \w \dd t 
\end{align}

Vergleicht man dies mit den Maxwell-Gleichungen, fällt auf, dass:

\begin{align}
\star \dd_s \star E &= \varrho \qquad\qquad,\varrho=\text{Ladungsdichte} \\
\star \dd_s\star B - \partial_t E &= j \qquad\qquad ,j=j_x \dd x + j_y \dd y + j_z \dd z
\end{align}

Definiert man den Strom $J=j - \varrho \  \dd t$, lassen sich (17) und (18) in 

\begin{align}
\dd \star F = \star J
\end{align}

zusammenfassen. \\

Insgesamt lässt sich also die gesamte Elektrodynamik auf die beiden gefundenen, äußerst eleganten Gleichungen zurückführen.

\begin{mybox}{Maxwell-Gleichungen}
\begin{align*}
\dd F &= 0 \\
\dd \star F &= \star J
\end{align*}
\end{mybox}
